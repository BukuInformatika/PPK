Penilaian tugas pokok dilakukan setiap satu minggu sekali dengan mengakumulasikan jumlah pekerjaan per-parameter selama 1 minggu.

\section{Dedikasi}

Menurut KBBI, Dedikasi bisa diartikan sebagai suatu pengorbanan tenaga, pikiran, dan waktu demi keberhasilan suatu usaha atau tujuan yang mulia. Dalam kata lain dedikasi juga bisa diartikan sebagai pengabdian. Pengambdian atau dedikasi yang bisa dilakukan mahasiswa D4 Teknik Informatika yaitu dengan melakukan pembuatan ataupun pembaharuan modul ajar di \textbf{\textit{https://github.com/bukuinformatika}}.

Adapun parameter penilaian dedikasi dapat dilihat pada tabel \ref{tab:nilaidedikasi}.

\begin{table}[h]
\caption{Penilaian Dedikasi}
\centering
\begin{tabular}{|c|c|}
\hline
Penilaian&Keterangan\\
\hline
Tinggi (3)&Full commit selama 5 hari kerja\\
\hline
Sedang (2)&Commit selama 4 hari kerja\\
\hline
Rendah (1)&Commit kurang dari 4 hari kerja\\
\hline
\end{tabular}
\label{tab:nilaidedikasi}
\end{table}

\section{Produktifitas}

Adapun parameter penilaian produktifitas dapat dilihat pada tabel \ref{tab:nilaiproduktifitas}.

\begin{table}[h]
\caption{Penilaian Produktifitas}
\centering
\begin{tabular}{|c|c|}
\hline
Penilaian&Keterangan\\
\hline
Tinggi (3)&Mengerjakan 5 pekerjaan harian\\
\hline
Sedang (2)&Mengerjakan 4 pekerjaan harian\\
\hline
Rendah (1)&Mengerjakan kurang dari 4 pekerjaan harian\\
\hline
\end{tabular}
\label{tab:nilaiproduktifitas}
\end{table}

\section{Integritas}

Adapun parameter penilaian integritas dapat dilihat pada tabel \ref{tab:nilaiintegritas}.

\begin{table}[h]
\caption{Penilaian Integritas}
\centering
\begin{tabular}{|c|c|}
\hline
Penilaian&Keterangan\\
\hline
Tinggi (3)&Tidak ada penolakan pull request\\
\hline
Sedang (2)&Ada 1 penolakan pull request\\
\hline
Rendah (1)&Ada lebih dari 1 penolakan pull request\\
\hline
\end{tabular}
\label{tab:nilaiintegritas}
\end{table}

\section{Disiplin}

Adapun parameter penilaian disiplin dapat dilihat pada tabel \ref{tab:nilaidisiplin}.

\begin{table}[h]
\caption{Penilaian Disiplin}
\centering
\begin{tabular}{|c|c|}
\hline
Penilaian&Keterangan\\
\hline
Tinggi (3)&Datang pulang sesuai jadwal selama 5 hari kerja\\
\hline
Sedang (2)&Ada 1 hari terlambat ataupun tidak masuk\\
\hline
Rendah (1)&Ada lebih dari 1 hari terlambat ataupun tidak masuk\\
\hline
\end{tabular}
\label{tab:nilaidisiplin}
\end{table}

\section{Loyalitas}

Adapun parameter penilaian loyalitas dapat dilihat pada tabel \ref{tab:nilailoyalitas}.

\begin{table}[h]
\caption{Penilaian Loyalitas}
\centering
\begin{tabular}{|c|c|}
\hline
Penilaian&Keterangan\\
\hline
Tinggi (3)&Area kerja bersih tanpa debu selama 5 hari kerja\\
\hline
Sedang (2)&Ada 1 hari area kerja tidak bersih\\
\hline
Rendah (1)&Ada lebih dari 1 hari area kerja tidak bersih\\
\hline
\end{tabular}
\label{tab:nilailoyalitas}
\end{table}

\section{Kreatif dan Inisiatif}

Adapun parameter penilaian kreatif dan inisiatif dapat dilihat pada tabel \ref{tab:nilaikreatifinisiatif}.

\begin{table}[h]
\caption{Penilaian Kreatif dan Inisiatif}
\centering
\begin{tabular}{|c|c|}
\hline
Penilaian&Keterangan\\
\hline
Tinggi (3)&Menjadi tentor dengan jumlah peserta minimal 10 orang\\
\hline
Sedang (2)&Menjadi tentor dengan jumlah peserta minimal 5 orang\\
\hline
Rendah (1)&Menjadi tentor dengan jumlah peserta kurang dari 5 orang\\
\hline
\end{tabular}
\label{tab:nilaikreatifinisiatif}
\end{table}

Kegiatan yang dilaksanakan merupakan kegiatan berbayar ataupun bersponsor.