Tugas Pokok merupakan pekerjaan yang dikerjakan selama \textit{Internship II} berlangsung. Pelaporan tugas ini dilakukan setiap satu hari sekali dengan menyertakan parameter dedikasi, produktifitas, interitas, disiplin, loyalitas, serta kreatif dan inisiatif. Untuk penilaian tugas pokok dilakukan setiap satu minggu sekali dengan mengakumulasikan jumlah pekerjaan per-parameter selama 1 minggu. Poin minimal untuk penilaian mingguan yaitu sebanyak 13 poin. Jika jumlah poin tidak mencapai poin minimal, maka mahasiswa harus menambah pekan untuk mengganti nilai yang kurang.

\section{Dedikasi}

Menurut KBBI, Dedikasi bisa diartikan sebagai suatu pengorbanan tenaga, pikiran, dan waktu demi keberhasilan suatu usaha atau tujuan yang mulia. Dalam kata lain dedikasi juga bisa diartikan sebagai pengabdian. Pengabdian atau dedikasi yang bisa dilakukan mahasiswa D4 Teknik Informatika yaitu dengan melakukan pembuatan ataupun pembaharuan modul ajar di \textbf{\textit{https://github.com/bukuinformatika}}.

Adapun parameter penilaian dedikasi dapat dilihat pada tabel \ref{tab:nilaidedikasi}.

\begin{table}[H]
\caption{Penilaian Dedikasi}
\centering
\begin{tabular}{|c|c|c|c|}
\hline
\textbf{No.}&\textbf{Label}&\textbf{Nilai}&\textbf{Keterangan}\\
\hline
1.&TINGGI&3&Full commit selama 5 hari kerja\\
\hline
2.&SEDANG&2&Commit selama 4 hari kerja\\
\hline
3.&RENDAH&1&Commit kurang dari 4 hari kerja\\
\hline
\end{tabular}
\label{tab:nilaidedikasi}
\end{table}

\section{Produktifitas}
Produktivitas mengandung arti sebagai perbandingan antara hasil yang dicapai (output) dengan keseluruhan sumber daya yang digunakan (input). Dengan kata lain bahwa produktivitas memliliki dua dimensi. Dimensi pertama adalah efektivitas yang mengarah kepada pencapaian target berkaitan dengan kuaitas, kuantitas dan waktu. Yang kedua yaitu efisiensi yang berkaitan dengan upaya membandingkan input dengan realisasi penggunaannya atau bagaimana pekerjaan tersebut dilaksanakan.

Adapun parameter penilaian produktifitas dapat dilihat pada tabel \ref{tab:nilaiproduktifitas}.

\begin{table}[H]
\caption{Penilaian Produktifitas}
\centering
\begin{tabular}{|c|c|c|c|}
\hline
\textbf{No.}&\textbf{Label}&\textbf{Nilai}&\textbf{Keterangan}\\
\hline
1.&TINGGI&3&Mengerjakan 5 pekerjaan harian\\
\hline
2.&SEDANG&2&Mengerjakan 4 pekerjaan harian\\
\hline
3.&RENDAH&1&Mengerjakan kurang dari 4 pekerjaan harian\\
\hline
\end{tabular}
\label{tab:nilaiproduktifitas}
\end{table}

\section{Integritas}
Integritas merupakan salah satu atribut terpenting/kunci yang harus dimiliki seorang pemimpin. Integritas adalah suatu konsep berkaitan dengan konsistensi dalam tindakan-tindakan, nilai-nilai, metode-metode, ukuran-ukuran, prinsip-prinsip, ekspektasi-ekspektasi dan berbagai hal yang dihasilkan.

Adapun parameter penilaian integritas dapat dilihat pada tabel \ref{tab:nilaiintegritas}.

\begin{table}[H]
\caption{Penilaian Integritas}
\centering
\begin{tabular}{|c|c|c|c|}
\hline
\textbf{No.}&\textbf{Label}&\textbf{Nilai}&\textbf{Keterangan}\\
\hline
1.&TINGGI&3&Tidak ada penolakan pull request\\
\hline
2.&SEDANG&2&Ada 1 penolakan pull request\\
\hline
3.&RENDAH&1&Ada lebih dari 1 penolakan pull request\\
\hline
\end{tabular}
\label{tab:nilaiintegritas}
\end{table}

Catatan:
\begin{itemize}
\item Selesaikan konflik terlebih dahulu, untuk menghindari penolakan saat pull request.
\end{itemize}

\section{Disiplin}
Disiplin merupakan perasaan taat dan patuh terhadap nilai-nilai yang dipercaya merupakan tanggung jawabnya. Dengan kata lain disiplin adalah patuh terhadap peraturan atau tunduk pada pengawasan dan pengendalian. Sedangkan pendisiplinan adalah usaha usaha untuk menanamkan nilai ataupun pemaksaan agar subjek memiliki kemampuan untuk menaati sebuah peraturan.

Adapun parameter penilaian disiplin dapat dilihat pada tabel \ref{tab:nilaidisiplin}.

\begin{table}[H]
\caption{Penilaian Disiplin}
\centering
\begin{tabular}{|c|c|c|c|}
\hline
\textbf{No.}&\textbf{Label}&\textbf{Nilai}&\textbf{Keterangan}\\
\hline
1.&TINGGI&3&Datang pulang sesuai jadwal selama 5 hari kerja\\
\hline
2.&SEDANG&2&Ada 1 hari terlambat ataupun tidak masuk\\
\hline
3.&RENDAH&1&Ada lebih dari 1 hari terlambat ataupun tidak masuk\\
\hline
\end{tabular}
\label{tab:nilaidisiplin}
\end{table}

\section{Loyalitas}
Loyalitas yang dimaksud yaitu usaha untuk menjaga kenyamanan dan keamanan lingkungan kerja. Contoh dari loyalitas diantaranya:
\begin{enumerate}
\item Menjaga area kerja tetap bersih dan bebas dari debu;
\item Menjaga kerapihan dan kenyaman area kerja;
\item Memperbaiki perangkat kerja yang rusak, dll.
\end{enumerate}
Adapun parameter penilaian loyalitas dapat dilihat pada tabel \ref{tab:nilailoyalitas}.

\begin{table}[H]
\caption{Penilaian Loyalitas}
\centering
\begin{tabular}{|c|c|c|c|}
\hline
\textbf{No.}&\textbf{Label}&\textbf{Nilai}&\textbf{Keterangan}\\
\hline
1.&TINGGI&3&Area kerja bersih tanpa debu selama 5 hari kerja\\
\hline
2.&SEDANG&2&Ada 1 hari area kerja tidak bersih\\
\hline
3.&RENDAH&1&Ada lebih dari 1 hari area kerja tidak bersih\\
\hline
\end{tabular}
\label{tab:nilailoyalitas}
\end{table}

\section{Kreatif dan Inisiatif}

Adapun parameter penilaian kreatif dan inisiatif dapat dilihat pada tabel \ref{tab:nilaikreatifinisiatif}.

\begin{table}[H]
\caption{Penilaian Kreatif dan Inisiatif}
\centering
\begin{tabular}{|c|c|c|c|}
\hline
\textbf{No.}&\textbf{Label}&\textbf{Nilai}&\textbf{Keterangan}\\
\hline
1.&TINGGI&3&Menjadi tentor dengan jumlah peserta minimal 10 orang\\
\hline
2.&SEDANG&2&Menjadi tentor dengan jumlah peserta minimal 5 orang\\
\hline
3.&RENDAH&1&Menjadi tentor dengan jumlah peserta kurang dari 5 orang\\
\hline
\end{tabular}
\label{tab:nilaikreatifinisiatif}
\end{table}

Catatan:
\begin{enumerate}
\item Kegiatan ini dilaksanakan minimal 1 kali selama Internship II berlangsung, dengan catatan target poin terpenuhi.
\item Kegiatan yang dilaksanakan merupakan kegiatan berbayar ataupun bersponsor.
\item Target poin yang dicapai minimal 3.
\item Jika poin tidak memenuhi target, maka silahkan untuk mengadakan kegiatan lagi sampai jumlah poin minimal terpenuhi.
\end{enumerate} 