\section{Menjaga Kebugaran Jasmani}
Seiring dengan dimulainya kegiatan Internship dan TA, tentunya pekerjaan yang akan kita lakukan akan semakin banyak dan menantang. Selain memikirkan tugas yang dikerjakan selama internship di Prodi dan IRC, kita juga harus dapat menyelesaikan penelitian Internship dan TA yang kita kerjakan. Tentunya kita harus dapat membagi waktu antara mengerjakan pekerjaan yang harus kita lakukan, waktu untuk kita mempelajari apa yg kita teliti, waktu untuk istrahat juga waktu untuk menjaga kebugaran yang kita miliki. 

Untuk menjaga kondisi kebugaran tubuh selama menjalani intership dan TA dan agar dapat berjalan lancar maka olahraga menjadi sebuah kewajiban. Oleh karena itu diberlakukan peraturan bagi peserta internship yang melakukan izin karena sakit, untuk wajib mengganti jumlah hari izin sakitnya pada hari sabtu atau minggu setelah dia sehat untuk melakukan olahraga di lapangan \textbf{gasibu} atau \textbf{sabuga} atau di \textbf{hutan siliwangi} sebanyak minimal 3 putaran pada jam 6.00 -08.00 pagi.

Catatan:
\begin{enumerate}
\item Kegiatan ini dilaksanakan bagi mahasiswa yang izin karena sakit.
\item Jika mahasiswa yang bersangkutan masih kurang sehat di hari sabtu atau minggu, mereka dapat menggantinya di sabtu-minggu depan.
\item Mahasiswa yang bersangkutan wajib mendokumentasikan kegiatan olahraga yang sudah mereka lakukan.
\item Kegiatan olaharaga bisa dilakukan di 3 tempat
\subitem Langan Sabuga ITB
\subitem Lapangan Gasibu
\subitem Hutan Siliwangi Bandung 
\end{enumerate} 